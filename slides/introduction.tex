\section{Introduction}

\begin{frame}{OpenGL Shader}

    What is shader?

    \begin{figure}
        \includegraphics[width=0.2\linewidth]{../resources/slides/opengl_logo.png}
        \includegraphics[width=0.2\linewidth]{../resources/slides/webgl_logo.png}
    \end{figure}

    \begin{figure}
        \includegraphics[width=0.15\linewidth]{../resources/slides/chrome_logo.png}
    \end{figure}

    \begin{figure}
        \includegraphics[width=0.2\linewidth]{../resources/slides/nvidia_logo.png}
        \includegraphics[width=0.2\linewidth]{../resources/slides/amd_logo.png}
    \end{figure}

\end{frame}

\begin{frame}{OpenGL Shader}

    What is the difference between GL shader and C?

    \begin{itemize}
        \item Language level support for vector and matrix operation
              \begin{itemize}
                  \item vec3(...) * mat3(...)
                  \item vec3 v; v.zyx = vec3(1, 2, 3);
              \end{itemize}
        \item No pointer
        \item No heap allocated memory (in the view of CPU memory structure)
        \item GPU based language
        \item Additional techniques for rendering pipeline (`in', `out', `texture', etc)
    \end{itemize}

\end{frame}


\begin{frame}{What We did and Why?}
    \begin{block}{What we did?}
        Compile a GPU language to LLVM IR
    \end{block}

    \begin{block}{Why we did it?}
        \begin{itemize}
            \item a toy language just for fun
            \item it may provide better debugging experience for GLSL
            \item Stack Overflow: \href{https://stackoverflow.com/questions/2508818/how-to-debug-a-glsl-shader}{How to debug a GLSL shader}.
            \begin{itemize}
                \item ``Using glslDevil or other tools is your best bet''
                \item ``Output something visually distinctive to the screen''    
            \end{itemize}
        \end{itemize}
    \end{block}

\end{frame}

\begin{frame}[fragile]{Sample Code}
   \begin{lstlisting}
#version 400
layout (location = 0) in vec3 inPos;
layout (location = 1) in vec3 inCol;
out vec3 vertCol;
uniform mat4 u_projectionMat44;
uniform mat4 u_viewMat44;
uniform mat4 u_modelMat44;
void main() {
    vertCol       = inCol;
    vec4 modelPos = u_modelMat44 * vec4( inPos, 1.0 );
    vec4 viewPos  = u_viewMat44 * modelPos;
    gl_Position   = u_projectionMat44 * viewPos;
}
   \end{lstlisting}
\end{frame}
