\subsection{Types}

\begin{frame}{Basic Types}
    Nearly the same as C and other imperative languages.
    \begin{itemize}
        \item bool
        \item int
        \item uint
        \item float
    \end{itemize}
\end{frame}

\begin{frame}{Vector and Matrix}
    \begin{itemize}
        \item bvecn, ivecn, uvecn, vecn
        \begin{itemize}
            \item Assuming that `v' has the type `vecn'
            \item v[x] (`float') returns the xth element of the vector
        \end{itemize}
        \item matnxm
        \begin{itemize}
            \item Assuming that `m' has the type `matnxm'
            \item m[0] (`vecn') returns the first column of the matrix
            \item m[x][y] (`float') returns the element located at xth column, yth row
        \end{itemize}
    \end{itemize}
\end{frame}

\begin{frame}{Swizzling}
    You can use x, y, z, or w, referring to the first, second, third, 
    and fourth components, respectively. Additionally, there are 3 sets 
    of swizzle masks. You can use \textbf{xyzw}, \textbf{rgba} (for colors), 
    or \textbf{stpq} (for texture coordinates).
\end{frame}

\begin{frame}[fragile]{Sample Code for Swizzling}
    \begin{lstlisting}
vec4 someVec;
someVec.x + someVec.y;
someVec.wzyx = vec4(1.0, 2.0, 3.0, 4.0);
someVec.zx = vec2(3.0, 5.0);
 
// void foo(inout vec3 someRef)
mat3 someMat;
someMat[1].bgr = someVec.xyz;
foo(someMat[0].zyx);
    \end{lstlisting}
\end{frame}

\begin{frame}[fragile]{Implementation for Swizzling: Reference Structure}
    In LLVM IR: [n x primitive\_type*]*
    \begin{lstlisting}
// normal reference: int a;
// i32* %a
store i32 %value, i32* %a

// vector reference: vec2 v;
// [2 x float*]* %v
%vxp = getelementptr [2 x float*], [2 x float*]* %v, i32 0, i32 0
%vx = load float*, float** %vxp
store float %value, %vx
    \end{lstlisting}
\end{frame}

\begin{frame}{Implementation for Swizzling: Reference Structure}
    Make modifications in the correct place. Take `.zyx' for an instance:
    \begin{figure}
        \includegraphics[width=0.45\linewidth]{../resources/slides/swizzle.png}
    \end{figure}
\end{frame}

\begin{frame}[fragile]{Constructor for vectors and matrices}
    \begin{lstlisting}
vec4(vec2(10.0, 11.0), 1.0, 3.5) == vec4(10.0, vec2(11.0, 1.0), 3.5);
vec3(vec4(1.0, 2.0, 3.0, 4.0)) == vec3(1.0, 2.0, 3.0);
vec4(vec3(1.0, 2.0, 3.0)); 
// error. Not enough components.
vec2(vec3(1.0, 2.0, 3.0)); 
// error. Too many components.
mat3 diagMatrix = mat3(5.0);
// Diagonal matrix with 5.0 on the diagonal.
    \end{lstlisting}
\end{frame}